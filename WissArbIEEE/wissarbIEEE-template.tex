%%%%%%%%%%%%%%%%%%%%%%%%%%%%%%%%%%%%%%%%%%%%%%%%%%%%%%%%%%%%%%%%%%%%%%%%%%%%%%%%
%2345678901234567890123456789012345678901234567890123456789012345678901234567890
%        1         2         3         4         5         6         7         8

%\documentclass[letterpaper, 10 pt, conference]{orbieeeconfpre}  % Comment this line out if you need a4paper

\documentclass[a4paper, 10pt, journal]{wissarbIEEE}      % Use this line for a4 paper
%\conference{IEEE Conference for Awesome ORB Research}

\bibliographystyle{orbref-num}

\IEEEoverridecommandlockouts                              % This command is only needed if 
                                                          % you want to use the \thanks command

\overrideIEEEmargins                                      % Needed to meet printer requirements.

% See the \addtolength command later in the file to balance the column lengths
% on the last page of the document

\usepackage{hyperref}
\usepackage{graphicx}
\usepackage{tabularx}
\usepackage{booktabs}
\usepackage{lipsum}
%von Max eingebunden:
%\usepackage{times}
\usepackage{url}
\usepackage{hyperref}
\usepackage{amsmath}              % Matheamtische Formeln
\usepackage{amsfonts}             % Mathematische Zeichensätze
\usepackage{amssymb}              % Mathematische Symbole
%\usepackage{underscore}

%% Wo sind die Bilder?
%\graphicspath{{bilder/}}


% Eigenes Makro für Bilder
\newcommand{\bild}[3]{
\begin{figure}[h]
\centering
  \includegraphics[width=#2]{#1}
  \caption{#3}
  \label{#1}
\end{figure}}

\newcommand{\length}[1]{\lvert \vec{#1} \rvert}

\title{\LARGE \bf
Automated conversion of CAD data in 3D low-poly models
}

\author{Maximilian Legnar and Prof. Dr. Rasenat}% <-this % stops a space

%\author{Maximilian Legnar$^{1}$ and Dr. Rasenat$^{2}$% <-this % stops a space
%
%\thanks{*Based on the guidelines published on the \href{http://conf.papercept.net/conferences/support/tex.php}{PaperCept conference manuscript management website}}% <-this % stops a space
%\thanks{$^{1}$Guybrush U. Threepwood is with the Institute for Pirate Sciences, Three-headed Monkey Group, University of  M\^el\'ee Island
%        {\tt\small gthreepwood@har.har-har.mi}}%
%\thanks{$^{2}$Alexander Schubert is with the Optimization in Robotics and Biomechanics Group, Institute of Computer Engineering, Heidelberg University, Berliner Str. 45, 69120 Heidelberg
%        {\tt\small alexander.schubert@ziti.uni-heidelberg.de, \href{http://orb.iwr.uni-heidelberg.de}{orb.iwr.uni-heidelberg.de}}}%
%}


\begin{document}

\maketitle

%%%%%%%%%%%%%%%%%%%%
\begin{abstract}

An automated conversion of CAD data into 3D low-poly models would offer great potential in the industrial environment. Automated transformation will allow companies to continue to use CAD models, which are already in existence. The converted models can be used, for example, in the field of virtual reality or even on mobile devices for high-performance and graphically appealing visualizations. This work addresses the question of how and in what quality such an automatic transformation is possible. By means of a prototypical implementation it is shown that such an automated conversion is possible and which problems occur.

\end{abstract}




%%%%%%%%%%%%%%%%%%%%%%%%%%%%%%%%%%%%%%%%%%%%%%%%%%%%%%%%%%%%%%%%%%%%%%%%%%%%%%%%%
%\section{The short paper document class}
%{\bf Important!} Please note that the \verb!wissarbIEEE.cls! file is to be used within the context of the lecture {\it Wissenschaftliches Arbeiten} at the University Heidelberg only and must not be distributed externally!
%
%%% Figure
%\begin{figure}[h]
%   \centering
%   \includegraphics[width=0.1\textwidth]{fig/knubbi.png}
%   \caption{Example picture.}
%   \label{fig:knubbi}
%\end{figure}
%
%\subsection{Bibliography styles}
%The \texttt{orbref-num.bst} bibliography style numbers the citations by their order of appearance. Here are two sample references: \cite{Newton1687,Mombaur2009}.
%
%\subsection{Lorem Ipsum}
%\lipsum[1]
%
%%% Table
%\begin{table}[h]
%\caption{Example table.}
%   \begin{tabularx}{0.48\textwidth}{llr}
%   		 \toprule 
%   		 Symbol & \multicolumn{1}{X}{Description} & Value [m] \\
%		 \midrule
%		  $A_x$ & Horizontal coordinate of A$^{*}$ & 0.0745 \\
%		  $A_z$ & Vertical coordinate of A$^{*}$ & 0.2650 \\  \noalign{\smallskip}
%		  
%		  $C_x$ & Horizontal coordinate of C$^{\#}$ & 0.0700 \\
%		  $C_z$ & Vertical coordinate of C$^{\#}$ & 0.1000 \\
%		  $D_x$ & Horizontal coordinate of D$^{\dag}$ & 0.0602 \\
%		  $D_z$ & Vertical coordinate of D$^{\dag}$ & 0.0860 \\
%		 \bottomrule
%   \end{tabularx}  \label{tab:initmodel}
%\end{table}
%
%\lipsum[2]

\bibliography{mybibfile}

\end{document}
